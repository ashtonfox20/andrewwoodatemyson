\documentclass[11pt]{article}

\usepackage{amsfonts}
%\usepackage{geometry}
\usepackage[paper=a4paper, 
            left=20.0mm, right=20.0mm, 
            top=25.0mm, bottom=25.0mm]{geometry}
\pagestyle{empty}
\usepackage{graphicx}
\usepackage{fancyhdr, lastpage, bbding, pmboxdraw}
\usepackage[usenames,dvipsnames]{color}
\definecolor{darkblue}{rgb}{0,0,.6}
\definecolor{darkred}{rgb}{.7,0,0}
\definecolor{darkgreen}{rgb}{0,.6,0}
\definecolor{red}{rgb}{.98,0,0}
\usepackage[colorlinks,pagebackref,pdfusetitle,urlcolor=darkblue,citecolor=darkblue,linkcolor=darkred,bookmarksnumbered,plainpages=false]{hyperref}
\renewcommand{\thefootnote}{\fnsymbol{footnote}}

\pagestyle{fancyplain}
\fancyhf{}
\lhead{ \fancyplain{}{Course Name} }
%\chead{ \fancyplain{}{} }
\rhead{ \fancyplain{}{\today} }
%\rfoot{\fancyplain{}{page \thepage\ of \pageref{LastPage}}}
\fancyfoot[RO, LE] {Page \thepage\ of \textcolor{black}{\pageref{LastPage}} }
\thispagestyle{plain}

%%%%%%%%%%%% LISTING %%%
\usepackage{listings}
\usepackage{caption}
\usepackage{subcaption}
\DeclareCaptionFont{white}{\color{white}}
\DeclareCaptionFormat{listing}{\colorbox{gray}{\parbox{\textwidth}{#1#2#3}}}
\captionsetup[lstlisting]{format=listing,labelfont=white,textfont=white}
\usepackage{verbatim} % used to display code
\usepackage{fancyvrb}
\usepackage{acronym}
\usepackage{amsthm, amsmath}
\usepackage{tikz}
    \usetikzlibrary{calc, arrows, arrows.meta, positioning}
\usepackage{amssymb,amsmath,stackengine}
\stackMath
\usepackage{ifthen}

\VerbatimFootnotes % Required, otherwise verbatim does not work in footnotes!

\definecolor{OliveGreen}{cmyk}{0.64,0,0.95,0.40}
\definecolor{CadetBlue}{cmyk}{0.62,0.57,0.23,0}
\definecolor{lightlightgray}{gray}{0.93}

\lstset{
	%language=bash,                          % Code langugage
	basicstyle=\ttfamily,                   % Code font, Examples: \footnotesize, \ttfamily
	keywordstyle=\color{OliveGreen},        % Keywords font ('*' = uppercase)
	commentstyle=\color{gray},              % Comments font
	numbers=left,                           % Line nums position
	numberstyle=\tiny,                      % Line-numbers fonts
	stepnumber=1,                           % Step between two line-numbers
	numbersep=5pt,                          % How far are line-numbers from code
	backgroundcolor=\color{lightlightgray}, % Choose background color
	frame=none,                             % A frame around the code
	tabsize=2,                              % Default tab size
	captionpos=t,                           % Caption-position = bottom
	breaklines=true,                        % Automatic line breaking?
	breakatwhitespace=false,                % Automatic breaks only at whitespace?
	showspaces=false,                       % Dont make spaces visible
	showtabs=false,                         % Dont make tabls visible
	columns=flexible,                       % Column format
	morekeywords={__global__, __device__},  % CUDA specific keywords
}

\newcommand{\question}[1]{\section*{\normalsize #1}}
% \newcommand{\mat}[1]{\begin{bmatrix}#1\end{bmatrix}}
% \newcommand{\extraspace}[]{
%     \begin{center}
%         \textbf{Use this page for extra space.}
%     \end{center}
% }


\DeclareMathOperator*{\argmax}{arg\,max}
\DeclareMathOperator*{\argmin}{arg\,min}
%\DeclareMathOperator*{\vec}[1]{\textbf{#1}}

\newcommand{\squig}{{\scriptstyle\sim\mkern-3.9mu}}
\newcommand{\lsquigend}{{\scriptstyle\lhd\mkern-3mu}}
\newcommand{\rsquigend}{{\scriptstyle\rule{.1ex}{0ex}\rhd}}
\newcounter{sqindex}
\newcommand\squigs[1]{%
  \setcounter{sqindex}{0}%
  \whiledo {\value{sqindex}< #1}{\addtocounter{sqindex}{1}\squig}%
}
\newcommand\rsquigarrow[2]{%
  \mathbin{\stackon[2pt]{\squigs{#2}\rsquigend}{\scriptscriptstyle\text{#1\,}}}%
}
\newcommand\lsquigarrow[2]{%
  \mathbin{\stackon[2pt]{\lsquigend\squigs{#2}}{\scriptscriptstyle\text{\,#1}}}%
}


\begin{document}
\begin{center}
    {\Large \textsc{Written Assignment 3}}
\end{center}
\begin{center}
    Due: Friday 03/22/2024 @ 11:59pm EST
\end{center}

\section*{\textbf{Disclaimer}}
I encourage you to work together, I am a firm believer that we are at our best (and learn better) when we communicate with our peers. Perspective is incredibly important when it comes to solving problems, and sometimes it takes talking to other humans (or rubber ducks in the case of programmers) to gain a perspective we normally would not be able to achieve on our own. The only thing I ask is that you report who you work with: this is \textbf{not} to punish anyone, but instead will help me figure out what topics I need to spend extra time on/who to help. When you turn in your solution (please use some form of typesetting: do \textbf{NOT} turn in handwritten solutions), please note who you worked with.\newline

\noindent Remember that if you have a partner, you and your partner should submit only \textbf{one} submission on gradescope.



\question{Question 1: Correctness of Alpha-Beta Pruning (25 points)}
Let $s$ be the state of the game, and assume that the game tree has a finite number of vertices. Let $v$ be the value produced by the minimax algorithm:
$$v = \texttt{Minimax}(s)$$

\noindent Let $v'$ be the result of running Alpha-Beta Pruning on $s$ with some initial values of $\alpha$ and $\beta$ (where $-\infty\le \alpha\le\beta\le+\infty$):
$$v' = \texttt{Alpha-Beta-Pruning}(s, \alpha, \beta)$$

\noindent Prove that the following statements are true:
\begin{itemize}
    \item If $\alpha \le v \le \beta$ then $v' = v$
    \item If $v\le \alpha$ then $v'\le \alpha$
    \item If $v \ge \beta$ then $v'\ge \beta$
\end{itemize}

\noindent This means that if the true minimax value is between $\alpha$ and $\beta$, then Alpha-Beta pruning returns the correct value. However, if the tru minimax value if outside of this range, then Alpha-Beta pruning may return a different value. However, the incorrect value that Alpha-Beta pruning returns is bounded in the same manner that the true minimax value is (i.e. if the true minimax value is $\le \alpha$ then the value produced by Alpha-Beta pruning is also $\le \alpha$ and vice versa). Note that this implies that Alpha-Beta pruning will be correct with initial values of $(-\infty, +\infty)$ for $(\alpha, \beta)$.\newline\newline

\noindent Hint: use induction. If $s$ is not a terminal state, then you can correctly assume that the claim above holds for all children of $s$. Use this assumption to prove that it also holds for $s$ (the base case is trivial: minimax and Alpha-Beta pruning produce the same value for terminal states)\newpage





\question{Question 2: CSP Reduction (25 points)}
Prove that any n-ary constraint can be converted into a set of binary constraints. Therefore, show that all CSPs can be converted into binary CSPs (and therefore we only need to worry about designing algorithms to process binary CSPs).\newline\newline\newline

\noindent Hint: When reducing a n-ary constraint: consider adding synthetic variable(s) (i.e. inventing new variables). Each synthetic variable should have a domain that comes from the cartesian product of the domains of the original variables involved in that constraint. We can then replace the original constraint with a set of binary constraints, where the original variables must match elements of the synthetic domain's value.
\newpage






\question{Extra Credit: Markov Blankets (50 points)}
A \textbf{Markov blanket} of Random Variable $X$ is the set of Random Variables that are the parents of $X$, the children of $X$, and the parents of $X$'s children. Prove that a Random Variable is independent of all other variables in a Bayesian network, given its Markov blanket and derive the following equation:
$$Pr[x'_i | mb(X_i)] = \alpha Pr[x'_i | Parents(X_i)]\prod\limits_{Y_j\in Children(X_i)} Pr[y_j | Parents(Y_j)]$$
where $mb(X)$ denotes the Markov blanket of Random Variable $X$.



\end{document}

